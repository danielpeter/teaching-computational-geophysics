\documentclass[11pt,titlepage,fleqn]{article}

\usepackage{amsmath}
\usepackage{amssymb}
\usepackage{latexsym}
\usepackage[round]{natbib}
\usepackage{xspace}
\usepackage{epsfig}
\usepackage{graphicx}
\usepackage{bm}
\usepackage{enumerate}

%%%%%%%%%%%%%%%%%%%%%%%%%%%%%%%%%%%%%%%%%%%%%%%%%%%%%%%%%%%%%
%       SPACING COMMANDS (Latex Companion, p. 52)
%%%%%%%%%%%%%%%%%%%%%%%%%%%%%%%%%%%%%%%%%%%%%%%%%%%%%%%%%%%%%

% page spacing
\usepackage{setspace}
\renewcommand{\baselinestretch}{1.5}
\textwidth 444pt \textheight 660pt
\oddsidemargin 15pt \evensidemargin 15pt
\headsep -20pt \headheight 0pt \hoffset 0pt \voffset 0pt

% no text indent
\setlength\parindent{0pt}

% font sans serif
\renewcommand{\familydefault}{\sfdefault}

%%%%%%%%%%%%%%%%%%%%%%%%%%%%%%%%%%%%%%%%%%%%%%%%%%%%%%%%%%%%%

\begin{document}

\begin{center}
\large {\bf Computational Geophysics - Homework **}\\
\today
\end{center}

\vspace{3cm}

Just a LaTex example file to get you started writing for scientific publications. \\

Put your report here...

% For example:
% The discretization of the problem is similar to Homework 1:
% \begin{eqnarray*}
% P^n_{i,j} \equiv P(x=i \times \Delta x, y=j \times \Delta y, t=n \times dt)
% \end{eqnarray*}
% ..
% Note that in the code I choose $\Delta x=\Delta y$ for simplicity.
% ..
%
% \section*{Simulation Results}
% \includegraphics[width=1.\textwidth]{./figures-homogeneous/6.jpg}
% ..

\end{document}
